\section{Foundations of Data Flow Analysis}



We saw a lot of examples of data flow analysis, eg. reaching definitions etc. Although 
there were differences between differeent types of data flow analysis, they did share number of 
things in common. Our goal is to develop a general purpose data flow analysis.




































Having shown several useful examples of the data-flow abstraction, 
we now study the family of data-flow schemas as a whole, abstractly. 
We shall answer several basic questions about data-flow algorithms formally:

\begin{itemize}

\item Under what circumstances is the iterative algorithm used in data-flow analysis correct?
\item How precise is the solution obtained by the iterative algorithm?
\item Will the iterative algorithm converge?
\item How fast is the convergence?
\end{itemize}


\subsection{Partial Order}\footnote{Based on \url{https://pages.cs.wisc.edu/~horwitz/CS704-NOTES/2.DATAFLOW.html}}

A binary relation R on a set S is called a partial ordering(poset), or partial order if and only if it is:

\begin{itemize}
\item \textbf{Reflexive} \(x \leq x\)
\item \textbf{Antisymmetric} if \(x \leq y\) and \(y \leq x\) then \(x = y\)
\item \textbf{Transitive} if \(x \leq y\) and \(y \leq z\) then \(x \leq z\)
\end{itemize} 



\subsection{Lattices}

A lattice is a poset in which every pair of elements has:

\begin{itemize}
\item a Least Upper Bound (the join of the two elements), and
\item a Greatest Lower Bound (the meet of the two elements).
\end{itemize}    



\subsection{Complete lattices}


A complete lattice is a lattice in which all subsets have a greatest lower bound 
and a least upper bound (the bounds must be in the lattice, but not necessarily 
in the subsets themselves). Note that Every finite lattice (i.e., S is finite) is complete.


\subsection{Monotonic and distributive functions}

A function f: L → L (where L is a lattice) is monotonic iff for all x,y in L: x ⊆ y implies f(x) ⊆ f(y).

A function f: L → L (where L is a lattice) is distributive iff for all x,y in L: f(x meet y) = f(x) meet f(y).

Every distributive function is also monotonic (proving that could be good practice!) but not vice versa. For the GEN/KILL dataflow problems, all dataflow functions are distributive. For constant propagation, all functions are monotonic, but not all functions are distributive.


\subsection{Fixed points}

x is a fixed point of function f iff f(x) = x.

\subsection{Meet Operator}


